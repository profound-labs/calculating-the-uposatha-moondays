% Created 2014-10-07 Tue 02:18
\documentclass[11pt,oneside]{memoir-article}
\usepackage{calculating-the-uposatha-moondays}
\renewcommand{\docVersion}{v0.1}
\renewcommand{\docUrl}{\href{https://github.com/profound-labs/calculating-the-uposatha-moondays/}{link}}
\hypersetup{ pdfauthor={Gambhiro Bhikkhu}, }
\author{Gambhiro Bhikkhu}
\date{\today}
\title{Calculating the Uposatha Moondays}
\hypersetup{
  pdfkeywords={},
  pdfsubject={},
  pdfcreator={Emacs 24.4.50.1 (Org mode 8.2.7c)}}
\begin{document}

\maketitle
\begin{tldr}
\begin{itemize}
\item The aim here is to describe the practical steps to calculate the
uposatha Full- and New Moon days, and indicate the astronomical
cycles that underlie the method.
\item Alternate 30 and 29 day lunar months, 12 months make one year. Add
an extra month 7 times in every 19 years, add an extra day 11 times
in every 57 years.
\item Conventions on how to practise this can differ by countries and
groups, resulting in consistent but different calendars.
\item Unforeseen adjustments to the predicted uposatha days in the
published Royal Thai Calendar can be expected.
\end{itemize}
\end{tldr}

\begin{quote}
Much appreciation for the answers from the Venerable Ajahns who
endured my questions and described their experience.

\textbf{Unresolved Questions:}

\begin{itemize}
\item The formulas predict 2016 to have an adhikavāra, see \ref{adhikavara-prediction}
\item Where in the year is the adhikavāra inserted?
\item Only Mahānikāya adds adhikavāra?
\item Where is the Pāli method in sec.\ref{pali-method} used?
\item Thai text in \href{https://github.com/profound-labs/calculating-the-uposatha-moondays/tree/master/references}{references folder}: Calendar and Era use in Thailand
from \emph{The Journal of the Royal Institute of Thailand} -- does it
have relevant notes?
\end{itemize}

This document is work-in-progress and represents only what I've been
able to find description of. See the Bibliography at sec.\ref{sec-4}
or \href{https://github.com/profound-labs/calculating-the-uposatha-moondays/archive/master.zip}{download a .zip} of this document with sources.

Comments, corrections or further information would be greatly
appreciated:

\texttt{Gambhiro Bhikkhu <gambhiro.bhikkhu.85@gmail.com>}
\end{quote}

\clearpage

\chapter{Thailand, Mahānikāya Method}
\label{sec-1}
\section{Alternate 30 and 29 day months}
\label{sec-1-1}

Counting from the last Full Moon of the previous lunar year (which may
be in January), the first month is 30 days, the second is 29 days:

\begin{center}
\begin{tabular}{lll}
15 days & \GaNewmoon{} New Moon & First uposatha of the Cold Season\\
15 days & \GaFullmoon{} Full Moon & End of first month, 30 days\\
15 days & \GaNewmoon{} New Moon & \\
14 days & \GaFullmoon{} Full Moon & End of second month, 29 days\\
\end{tabular}
\end{center}

The \GaWaxingmoon{} Waxing- and \GaWaningmoon{} Waning Moons are on the 8th day.

\includegraphics[width=\linewidth]{two-months.pdf}

Keep alternating 30 and 29 day months. One season is four months, one
year is three seasons: Cold-, Hot- and Rainy Season. In a year with
nothing special, the calendar is finished. See Table \ref{tbl-month-names}
for the Pāli names of months and seasons.

In some years an extra month (adhikamāsa) or an extra day
(adhikavāra) has to be added.

\section{Adding the extra month}
\label{sec-1-2}

The extra month (adhikamāsa) is added 7 times in every 19 year, in a repeating
pattern of 3-3-2 - 3-3-3-2 years. This is a shorthand for the formulas
at \ref{fig-suriyayatra} which generate this pattern. Table
\ref{tbl-cycle-adhikamasa} shows adhikamāsa years for 1996-2034.

In Thai practice, the extra month is a 30 day month inserted after the
8th month (\emph{Āsāḷha}), at the end of the Hot Season. The convention is
to call this the `second 8th' or `second \emph{Āsāḷha}', marked as 8/8.

Vassa starts after the 2nd Āsāḷha, on the day after the Full Moon
uposatha of 8/8.

\begin{center}
\begin{tabular}{rlr}
order & name & days\\
\hline
8 & Āsāḷha & 29\\
8/8 & 2nd Āsāḷha & 30\\
9 & Savaṇa & 30\\
\end{tabular}
\end{center}
\section{Adding the extra day}
\label{sec-1-3}
\label{adding-extra-day}

The extra day (adhikavāra) is added 11 times in every 57 year.

\subsection{Checkme: Adhikavāra prediction}
\label{sec-1-3-1}
\label{adhikavara-prediction}

The formulas predict 2016 to have an
adhikavāra. See below for the
\emph{kammacubala} (K), \emph{avoman} (A) and \emph{thaloengsok} (T) values produced
with the formulas \ref{fig-suriyayatra}.

See description at sec.\ref{adhikamat-years} and
sec.\ref{adhikawan-years}.

2015 qualifies for adhikamat, but also for adhikawan, and so the
adhikawan would be carried on to 2016.

\begin{center}
\begin{tabular}{rrlrrr}
AD & CS & type & K & A & T\\
2014 & 1376 & d & 395 & 137 & 17\\
2015 & 1377 & m & 188 & 0 & 28\\
2016 & 1378 & d & 781 & 566 & 9\\
\end{tabular}
\end{center}

Also weird because two years in succession (2014 and 2015) would
qualify for adhikawan.

2014 is \emph{edge-case}, having avoman 137, perhaps the condition is $A <
137$ and not $A <= 137$? Possible, if the interpretation of the rule
is to index day values from 0, not from 1.

More past calendars with adhikawan are needed to look-up and formalize
testing.

\section{Major Moondays}
\label{sec-1-4}

Buddhist communities observe key annual events on the Full Moon
days of four lunar months:

\begin{center}
\begin{tabular}{lll}
 & Lunar Month & \\
Māgha Pūjā & 3rd & \\
Visākha Pūjā & 6th & \\
Āsāḷha Pūjā & 8th & Entering Vassa on the next day\\
Assayuja Pūjā & 11th & Pavāraṇā Day, the end of Vassa\\
\end{tabular}
\end{center}

The Full Moon day is on the last day of a given month. The next month
starts on the following day (first day of the waning phase), thus the
first uposatha will be on a New Moon.

\chapter{Adding the extra month, Pāli method}
\label{sec-2}
\label{pali-method}

\emph{The following is adapted from Ajahn Khemanando for recent
years.}\cite{khemanando-adhikamasa}

Table \ref{tbl-cycle-adhikamasa} shows the 19-year cycle between
1996-2034.

\begin{table}[p]
\caption{\label{tbl-cycle-adhikamasa} Adhikamāsa years for 1996-2034 and inserting the extra month according to Thai and Pāli method.}\legend{\Delta m for years since last adhikamāsa.}
\centering
\begin{tabular}{rrrrl|rlrr}
$\Delta$ m &  &  &  & Month & Month & Season & New & Full\\
 &  &  &  & (Thai) & (Pāli) &  &  & \\
\hline
 & 0 & 1996 & 2015 & 8/8 & 8 & Rainy & 8 & 12\\
 & 1 &  &  &  &  &  &  & \\
 & 2 &  &  &  &  &  &  & \\
3 & 3 & 1999 & 2018 & 8/8 & 5 & Hot & 4 & 8/8\\
 & 4 &  &  &  &  &  &  & \\
 & 5 &  &  &  & 2 & Cold & 12 & 5\\
3 & 6 & 2002 & 2021 & 8/8 &  & Cold & 12 & 5\\
 & 7 &  &  &  &  &  &  & \\
2 & 8 & 2004 & 2023 & 8/8 & 10 & Rainy & 8 & 12\\
 & 9 &  &  &  &  &  &  & \\
 & 10 &  &  &  &  &  &  & \\
3 & 11 & 2007 & 2026 & 8/8 & 7 & Hot & 4 & 8/8\\
 & 12 &  &  &  &  &  &  & \\
 & 13 &  &  &  & 3 & Cold & 12 & 5\\
3 & 14 & 2010 & 2029 & 8/8 &  & Cold & 12 & 5\\
 & 15 &  &  &  &  &  &  & \\
 & 16 &  &  &  & 12 & Cold & 12 & 5\\
3 & 17 & 2013 & 2032 & 8/8 &  & Cold & 12 & 5\\
 & 18 &  &  &  &  &  &  & \\
2 & 19 & 2015 & 2034 & 8/8 & 8 & Rainy & 8 & 12\\
\end{tabular}
\end{table}

\begin{description}
\item[{$\Delta$ m:}] years science the last adhikamāsa
\item[{Month:}] the Thai lunar month into which the adhikamāsa is inserted
\item[{Season:}] the season in which the adhikamāsa fall in that
particular year
\item[{New and Full:}] the first and last uposatha of the 5-month season
in which the adhikamāsa falls, numbered in Thai
lunar months
\end{description}

If the adhikamāsa falls on the 2nd, 3rd, or 12th Thai lunar month,
there will be \emph{two} 8th months (8 and 8/8) the following year.

E.g. In 2001, the adhikamāsa comes as the 2nd lunar month in the
Cold Season, so the following year, 2002, has two 8th months (8 and
8/8). There will thus be \emph{ten} uposathas in the Cold Season, the
first being the New Moon of the 12th Thai lunar month (2001) and the
last being the Full Moon of the 5th Thai lunar month, 2002.

\clearpage

\chapter{The Thai luni-solar calendar}
\label{sec-3}

Luni-solar calendars are constructed so to count years according to
the \emph{solar} cycle, but to count months according to the \emph{lunar} cycle.

\begin{center}
\begin{tabular}{ll}
tropical year\footnotemark of the Earth & 365.24219 days\\
synodic month\footnotemark of the Moon & \textasciitilde{}29.53 days, can vary up to 7 hours\\
\end{tabular}
\end{center}\footnotetext[1]{tropical year: the time it takes the Earth to
complete an orbit around the Sun}\footnotetext[2]{synodic month: the time it takes the Moon to reach
the same visual phase}

The epoch of the Thai calendar is 25 March 638 AD.

The Thai luni-solar calendar is \emph{procedural}, it uses a few constant,
key numbers derived from astronomical observations, and applies a
series of mechanical calculations (i.e. the ``rules'') again and again
to generate the dates of lunar phases and new years.

\begin{quote}
This working is deliberately concise, since it thereby reflects how
the calculation would have been made by a South East Asian calendrist.
Each stage is subjected to an operation learnt by rote, and the
underlying theory disappears from view. The rote operations, however,
will provide a valid answer for any date in any year. It seemed
greatly preferable to set out the procedure thus starkly, rather than
to give a detailed exposition of what is involved.\cite{eade-interpolation}
\end{quote}

Southeast Asian astronomers refined a fraction to obtain the length of
the year:

\begin{equation}
\frac{292207}{800} = 365.25875\ \text{days}\cite{eade-interpolation}
\end{equation}

This is 0.01656 days longer than the modern measurement (accumulating
1 day in \textasciitilde{}60 years). Remarkably, the \emph{suriyayatra} accounts for this
and generates accurate results:

\begin{quote}
For instance, a Pagan inscription of 14 April 1288 AD maintains that
at midnight the Sun's position was 0 signs, 19 degrees and 59 minutes:
the computer program returns
0~19~59.\cite{eade-calendrical}
\end{quote}

Nonetheless, the calendar dates published in Thailand (historical or
recent) in a given year reflect not only these principles, but also
additional adjustments which cannot be foreseen or retraced.

\begin{quote}
The historical record however, frequently defies prediction, forcing
the conclusion that the pressure upon the \emph{horas} (astronomers /
astrologers) was not to follow the ``rules'' but merely, within some
more leisurely constraints, to ensure that the calendar did not get
out of control.\cite{eade-calendrical}
\end{quote}

\clearpage

\section{Year Types}
\label{sec-3-1}

\begin{multicols}{2}

We are concerned with three types of calendar years:

\begin{description}
\item[{Cal A}] Normal with 354 days
\item[{Cal B}] Adhikavāra with 355 days
\item[{Cal C}] Adhikamāsa with 384 days
\end{description}

\columnbreak

Comparing these to normal and solar leap years:

\begin{center}
\begin{tabular}{lrrr}
 & A & B & C\\
Lunar & 354 & 355 & 384\\
Solar & 365 & 365 & 365\\
difference & +11 & +10 & -19\\
\hline
 & A & B & C\\
Lunar & 354 & 355 & 384\\
Solar Leap & 366 & 366 & 366\\
difference & +12 & +11 & -18\\
\end{tabular}
\end{center}

\end{multicols}

\section{Adhikamat years}
\label{sec-3-2}
\label{adhikamat-years}

The \emph{suriyayatra} principle to determine adhikamat years is:

\begin{quote}
If the day of \emph{thaloengsok} (astronomical New Year)
lies either within 25 to 29 (in Citta-māsa) or 1 to 5 (in
Visākha-māsa), then the year is adhikamat.\cite{prasert-ngan}
\end{quote}

The \emph{thaloengsok} is the value of T in Figure \ref{fig-suriyayatra}.

\section{Adhikawan years}
\label{sec-3-3}
\label{adhikawan-years}

\begin{quote}
Two components of the \emph{suriyayatra} are known as the \emph{kammacubala} and
the \emph{avoman}, and it is the values of these two elements at the start
of the year that determine the matter:

\begin{itemize}
\item if the kammacubala value is 207 or less, then the year is leap year
\item in a leap year, if the avoman is 126 or less, the year will have an
extra day
\item in a normal year, if the avoman is 137 or less, the year will have
and extra day\cite{eade-interpolation}
\end{itemize}
\end{quote}

The \emph{kammacubala} and \emph{avoman} are the value of K and A in Figure
\ref{fig-suriyayatra}.

In Thailand, years with an extra month are not allowed to also have an
extra day, and the adhikawan will be assigned to the next year.

\section{Suriyayatra formulas}
\label{sec-3-4}

See Figure \ref{fig-suriyayatra}.

\begin{figure}
\caption{\label{fig-suriyayatra}Finding astronomical values with the \emph{suriyayatra} calculation\cite{eade-interpolation}}
\legend{Start with Y, the given Common Era year. Significant values are assigned names. K for \emph{kammacubala}, A for \emph{avoman}, T for \emph{thaloengsok} (the New Year).}
\begin{eqnarray}
a & = & ((Y - 638) * 292207) + 373 \\
h & = & \lfloor a/800 + 1 \rfloor \\
K & = & 800 - (a \bmod 800) \\
A & = & ((h*11) + 650) \bmod 692 \\
b & = & \lfloor ((h*11) + 650) / 692 \rfloor \\
T & = & (b + h) \bmod 30
\end{eqnarray}
\end{figure}

\begin{table}[p]
\caption{Adhikamat and adhikawan in the period 1958 to 1978 (CS 1320-1340).\cite{eade-interpolation}}\legend{m for adhikamat, d for adhikawan years, \Delta m and \Delta d for years since last adhikamat and adhikawan.}
\centering
\begin{tabular}{rrrrrlrr}
 & $\Delta$ d &  & $\Delta$ m & year & type & Asalha & 2nd Asalha\\
\hline
 &  & 0 &  & 1320 & m & 19:42 & 22:24\\
0 &  & 1 &  & 1321 & d & 21:05 & \\
1 &  & 2 &  & 1322 &  & 20:40 & \\
2 &  & 3 & 3 & 1323 & m & 19:12 & 22:00\\
3 &  & 4 &  & 1324 &  & 20:38 & \\
4 & 4 & 5 &  & 1325 & d & 19:34 & \\
5 &  & 6 & 3 & 1326 & m & 19:38 & 22:05\\
6 &  & 7 &  & 1327 &  & 21:15 & \\
7 &  & 8 & 2 & 1328 & m & 19:20 & 22:55\\
8 &  & 9 &  & 1329 &  & 21:48 & \\
9 & 5 & 10 &  & 1330 & d & 20:26 & \\
10 &  & 11 & 3 & 1331 & m & 19:59 & 22:50\\
11 &  & 12 &  & 1332 &  & 21:20 & \\
12 &  & 13 &  & 1333 &  & 20:02 & \\
13 &  & 14 & 3 & 1334 & m & 19:03 & 21:33\\
14 & 5 & 15 &  & 1335 & d & 20:40 & \\
15 &  & 16 &  & 1336 &  & 20:44 & \\
16 &  & 17 & 3 & 1337 & m & 19:44 & 22:19\\
17 &  & 18 &  & 1338 &  & 21:11 & \\
18 &  & 19 & 2 & 1339 & m & 19:45 & 22:35\\
19 & 5 &  &  & 1340 & d & 21:05 & \\
\end{tabular}
\end{table}

\clearpage
\section{Names of the months}
\label{sec-3-5}

The name of a given month is determined by the astrological sign which
the Full Moon enters at midnight. See Table \ref{tbl-month-names}.

\begin{table}[htb]
\caption{\label{tbl-month-names}Lunar and Solar Months and Zodiacs\cite{hasapannyo-zodiac}}\legend{\mA{} marks 29 day months having a 14 day New Moon (\emph{amāvasī cātuddasī}).}
\centering
\begin{tabular}{llll}
Season & Lunar Month & Solar Month & Solar Zodiac\\
 &  &  & (Western / Sanskrit)\\
\hline
Hemanta-utu & Magasira-māsa & December & Sagittarius / Dhanus\\
Cold Season & Phussa-māsa\mA & January & Capricorn / Makara\\
 & Māgha-māsa & February & Aquarius / Kumbha\\
 & Phagguṇa-māsa\mA & March & Pisces / Mīna\\
\hline
Gimha-utu & Citta-māsa & April & Aries / Meṣa\\
Hot Season & Visākha-māsa\mA & May & Taurus / Vṛṣabha\\
 & Jeṭṭha-māsa & June & Gemini / Mithuna\\
 & Āsāḷha-māsa\mA & July & Cancer / Karkaṭa\\
\hline
Vassāna-utu & Savaṇa-māsa & August & Leo / Siṃha\\
Rainy Season & Bhaddapāda-māsa\mA & September & Virgo / Kanyā\\
 & Assayuja-māsa & October & Libra / Tulā\\
 & Kattika-māsa\mA & November & Scorpio / Vṛścika\\
\end{tabular}
\end{table}

\backmatter

\chapter{Bibliography}
\label{sec-4}
\label{bibliography}

\bibliographystyle{plain}
\bibliography{bibentries}
\chapter{Colophon}
\label{sec-5}

\href{http://orgmode.org/}{Org-mode} and \LaTeX. Sources at \href{https://github.com/profound-labs/calculating-the-uposatha-moondays/}{Github}.

Comments, corrections or further information would be greatly
appreciated.

\texttt{Gambhiro Bhikkhu <gambhiro.bhikkhu.85@gmail.com>}

Last updated on 2014-10-07.
% Emacs 24.4.50.1 (Org mode 8.2.7c)
\end{document}
